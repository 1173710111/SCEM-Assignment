% Options for packages loaded elsewhere
\PassOptionsToPackage{unicode}{hyperref}
\PassOptionsToPackage{hyphens}{url}
%
\documentclass[
]{article}
\usepackage{amsmath,amssymb}
\usepackage{iftex}
\ifPDFTeX
  \usepackage[T1]{fontenc}
  \usepackage[utf8]{inputenc}
  \usepackage{textcomp} % provide euro and other symbols
\else % if luatex or xetex
  \usepackage{unicode-math} % this also loads fontspec
  \defaultfontfeatures{Scale=MatchLowercase}
  \defaultfontfeatures[\rmfamily]{Ligatures=TeX,Scale=1}
\fi
\usepackage{lmodern}
\ifPDFTeX\else
  % xetex/luatex font selection
\fi
% Use upquote if available, for straight quotes in verbatim environments
\IfFileExists{upquote.sty}{\usepackage{upquote}}{}
\IfFileExists{microtype.sty}{% use microtype if available
  \usepackage[]{microtype}
  \UseMicrotypeSet[protrusion]{basicmath} % disable protrusion for tt fonts
}{}
\makeatletter
\@ifundefined{KOMAClassName}{% if non-KOMA class
  \IfFileExists{parskip.sty}{%
    \usepackage{parskip}
  }{% else
    \setlength{\parindent}{0pt}
    \setlength{\parskip}{6pt plus 2pt minus 1pt}}
}{% if KOMA class
  \KOMAoptions{parskip=half}}
\makeatother
\usepackage{xcolor}
\usepackage[margin=1in]{geometry}
\usepackage{color}
\usepackage{fancyvrb}
\newcommand{\VerbBar}{|}
\newcommand{\VERB}{\Verb[commandchars=\\\{\}]}
\DefineVerbatimEnvironment{Highlighting}{Verbatim}{commandchars=\\\{\}}
% Add ',fontsize=\small' for more characters per line
\usepackage{framed}
\definecolor{shadecolor}{RGB}{248,248,248}
\newenvironment{Shaded}{\begin{snugshade}}{\end{snugshade}}
\newcommand{\AlertTok}[1]{\textcolor[rgb]{0.94,0.16,0.16}{#1}}
\newcommand{\AnnotationTok}[1]{\textcolor[rgb]{0.56,0.35,0.01}{\textbf{\textit{#1}}}}
\newcommand{\AttributeTok}[1]{\textcolor[rgb]{0.13,0.29,0.53}{#1}}
\newcommand{\BaseNTok}[1]{\textcolor[rgb]{0.00,0.00,0.81}{#1}}
\newcommand{\BuiltInTok}[1]{#1}
\newcommand{\CharTok}[1]{\textcolor[rgb]{0.31,0.60,0.02}{#1}}
\newcommand{\CommentTok}[1]{\textcolor[rgb]{0.56,0.35,0.01}{\textit{#1}}}
\newcommand{\CommentVarTok}[1]{\textcolor[rgb]{0.56,0.35,0.01}{\textbf{\textit{#1}}}}
\newcommand{\ConstantTok}[1]{\textcolor[rgb]{0.56,0.35,0.01}{#1}}
\newcommand{\ControlFlowTok}[1]{\textcolor[rgb]{0.13,0.29,0.53}{\textbf{#1}}}
\newcommand{\DataTypeTok}[1]{\textcolor[rgb]{0.13,0.29,0.53}{#1}}
\newcommand{\DecValTok}[1]{\textcolor[rgb]{0.00,0.00,0.81}{#1}}
\newcommand{\DocumentationTok}[1]{\textcolor[rgb]{0.56,0.35,0.01}{\textbf{\textit{#1}}}}
\newcommand{\ErrorTok}[1]{\textcolor[rgb]{0.64,0.00,0.00}{\textbf{#1}}}
\newcommand{\ExtensionTok}[1]{#1}
\newcommand{\FloatTok}[1]{\textcolor[rgb]{0.00,0.00,0.81}{#1}}
\newcommand{\FunctionTok}[1]{\textcolor[rgb]{0.13,0.29,0.53}{\textbf{#1}}}
\newcommand{\ImportTok}[1]{#1}
\newcommand{\InformationTok}[1]{\textcolor[rgb]{0.56,0.35,0.01}{\textbf{\textit{#1}}}}
\newcommand{\KeywordTok}[1]{\textcolor[rgb]{0.13,0.29,0.53}{\textbf{#1}}}
\newcommand{\NormalTok}[1]{#1}
\newcommand{\OperatorTok}[1]{\textcolor[rgb]{0.81,0.36,0.00}{\textbf{#1}}}
\newcommand{\OtherTok}[1]{\textcolor[rgb]{0.56,0.35,0.01}{#1}}
\newcommand{\PreprocessorTok}[1]{\textcolor[rgb]{0.56,0.35,0.01}{\textit{#1}}}
\newcommand{\RegionMarkerTok}[1]{#1}
\newcommand{\SpecialCharTok}[1]{\textcolor[rgb]{0.81,0.36,0.00}{\textbf{#1}}}
\newcommand{\SpecialStringTok}[1]{\textcolor[rgb]{0.31,0.60,0.02}{#1}}
\newcommand{\StringTok}[1]{\textcolor[rgb]{0.31,0.60,0.02}{#1}}
\newcommand{\VariableTok}[1]{\textcolor[rgb]{0.00,0.00,0.00}{#1}}
\newcommand{\VerbatimStringTok}[1]{\textcolor[rgb]{0.31,0.60,0.02}{#1}}
\newcommand{\WarningTok}[1]{\textcolor[rgb]{0.56,0.35,0.01}{\textbf{\textit{#1}}}}
\usepackage{graphicx}
\makeatletter
\def\maxwidth{\ifdim\Gin@nat@width>\linewidth\linewidth\else\Gin@nat@width\fi}
\def\maxheight{\ifdim\Gin@nat@height>\textheight\textheight\else\Gin@nat@height\fi}
\makeatother
% Scale images if necessary, so that they will not overflow the page
% margins by default, and it is still possible to overwrite the defaults
% using explicit options in \includegraphics[width, height, ...]{}
\setkeys{Gin}{width=\maxwidth,height=\maxheight,keepaspectratio}
% Set default figure placement to htbp
\makeatletter
\def\fps@figure{htbp}
\makeatother
\setlength{\emergencystretch}{3em} % prevent overfull lines
\providecommand{\tightlist}{%
  \setlength{\itemsep}{0pt}\setlength{\parskip}{0pt}}
\setcounter{secnumdepth}{-\maxdimen} % remove section numbering
\ifLuaTeX
  \usepackage{selnolig}  % disable illegal ligatures
\fi
\usepackage{bookmark}
\IfFileExists{xurl.sty}{\usepackage{xurl}}{} % add URL line breaks if available
\urlstyle{same}
\hypersetup{
  pdftitle={TaskB1},
  pdfauthor={zerofrom},
  hidelinks,
  pdfcreator={LaTeX via pandoc}}

\title{TaskB1}
\author{zerofrom}
\date{2024-11-24}

\begin{document}
\maketitle

\subsection{B.1}\label{b.1}

\subsubsection{(1) Value of a}\label{value-of-a}

Considering chat \(x\) follows the probability density function:

\[
p_{\lambda}(x)= \begin{cases}a e^{-\lambda(x-b)} & \text { if } x \geqslant b,  \tag{1}\\ 0 & \text { if } x<b,\end{cases}
\]

The probability density function \(p_{\lambda}(x)\) must follows:

\[
\begin{equation*}
\int_{-\infty}^{\infty} p_{\lambda}(x) d x=1 \tag{2}
\end{equation*}
\]

Since \(p_{\lambda}(x)=0\) when \(x<b\), it follows that:

\[
\begin{equation*}
\int_{b}^{\infty} p_{\lambda}(x) d x=1 \tag{3}
\end{equation*}
\]

Replacing expression (1) into expression (3) yields:

\[
\begin{equation*}
\int_{b}^{\infty} a e^{-\lambda(x-b)} d x=1 \tag{4}
\end{equation*}
\]

Assuming that \(u=x-b\), it can be obtained that:

\[
\begin{equation*}
x=u+b, d x=d u \tag{5}
\end{equation*}
\]

Assuming that \(x=b\), then \(u=0\), therefore the upper and lower
limits of integration become \(0 \rightarrow \infty\), substituting
expression \((s)\) :

\[
\begin{equation*}
\int_{0}^{\infty} a e^{-\lambda u} d u=1 \tag{6}
\end{equation*}
\]

Using the exponential integration formula:

\[
\begin{equation*}
\int_{0}^{\infty} e^{-a u} d u=\frac{1}{a} \tag{7}
\end{equation*}
\]

Thus:

\[
\begin{equation*}
a \cdot \frac{1}{\lambda}=1 \tag{8}
\end{equation*}
\]

The solution is:

\[
\begin{equation*}
a=\frac{1}{\lambda} \tag{9}
\end{equation*}
\]

\subsubsection{\texorpdfstring{In conclusion, the value of a is
\(\frac{1}{\lambda}\).}{In conclusion, the value of a is \textbackslash frac\{1\}\{\textbackslash lambda\}.}}\label{in-conclusion-the-value-of-a-is-frac1lambda.}

\paragraph{\texorpdfstring{(2) i) Population Mean \(E[X]\)
:}{(2) i) Population Mean E{[}X{]} :}}\label{i-population-mean-ex}

From the definition of the mean, there is:

\[
\begin{equation*}
E[x]=\int_{-\infty}^{\infty} x \cdot p_{\lambda}(x) d x \tag{1}
\end{equation*}
\]

Replacing \(p_{\lambda}(x)\) and \(a=\lambda\) :

\[
\begin{equation*}
E[x]=\int_{b}^{\infty} x \cdot \lambda e^{-\lambda(x-b)} d x \tag{2}
\end{equation*}
\]

Assuming \(u=x-b\), then \(x=u+b\), and \(d x=d u\) :

\[
\begin{align*}
E[X] &= \int_{0}^{\infty}(u + b) \cdot \lambda e^{-\lambda u} \, du \\
     &= \lambda \cdot \int_{0}^{\infty} u e^{-\lambda u} \, du + \lambda b \int_{0}^{\infty} e^{-\lambda u} \, du \tag{3}
\end{align*}
\]

Integral formula for exponential functions:

\[
\begin{equation*}
\int_{0}^{\infty} e^{-a x} d x=\frac{1}{a}, \int_{0}^{\infty} x e^{-a x} d x=\frac{1}{a^{2}} \tag{4}
\end{equation*}
\]

Replacing expression (t) into expression (3), there is:

\[
\begin{aligned}
E[x] & =\lambda \cdot \frac{1}{\lambda^{2}}+\lambda b \cdot \frac{1}{\lambda} \\
& =\frac{1}{\lambda}+b
\end{aligned}
\]

\subsubsection{\texorpdfstring{In conclusion, the population mean
\(E[x]=\frac{1}{\lambda}+b\).}{In conclusion, the population mean E{[}x{]}=\textbackslash frac\{1\}\{\textbackslash lambda\}+b.}}\label{in-conclusion-the-population-mean-exfrac1lambdab.}

\paragraph{\texorpdfstring{ii) Standard Deviation \(\sigma x\)
:}{ii) Standard Deviation \textbackslash sigma x :}}\label{ii-standard-deviation-sigma-x}

The standard deviation formula:

\[
\begin{equation*}
\sigma x=\sqrt{\operatorname{Var}(x)}=\sqrt{E\left[x^{2}\right]-(E[x])^{2}} \tag{1}
\end{equation*}
\]

From the derivation of \(E[x]\), it follows that:

\[
E[x]=\frac{1}{\lambda}+b \tag{2}
\]

The formula for \(E\left[x^{2}\right]\) is:

\[
\begin{equation*}
E\left[x^{2}\right]=\int_{b}^{\infty} x^{2} \cdot p_{\lambda}(x) d x \tag{3}
\end{equation*}
\]

Substituting \(P_{\lambda}(x)\) :

\[
\begin{equation*}
E\left[x^{2}\right]=\int_{b}^{\infty} x^{2} \cdot \lambda e^{-\lambda(x-b)} d x \tag{4}
\end{equation*}
\]

Assuming \(u=x-b\). then \(x=u+b\), and \(d x=d u\) :

\[
\begin{align*}
E\left[x^{2}\right]= & \int_{0}^{\infty}(u+b)^{2} \cdot \lambda e^{-\lambda u} d u \\
= & \lambda \int_{0}^{\infty}\left(u^{2}+2 b u+b^{2}\right) \cdot e^{-\lambda u} d u \\
= & \lambda \int_{0}^{\infty} u^{2} e^{-\lambda u} d u+2 \lambda b \int_{0}^{\infty} u e^{-\lambda u} d u  +\lambda b^{2} \int_{0}^{\infty} e^{-\lambda u} d u \quad  \tag{5}
\end{align*}
\]

Integrals formula for exponential functions:

\[
\begin{align*}
& \int_{0}^{\infty} e^{-a x} d x=\frac{1}{a} \\
& \int_{0}^{\infty} x e^{-a x} d x=\frac{1}{a^{2}} \\
& \int_{0}^{\infty} x^{2} e^{-a x} d x=\frac{2}{a^{3}} \tag{6}
\end{align*}
\]

Replacing formulas (6) into expression (5), there is:

\[
\begin{align*}
E\left[x^{2}\right] & =\lambda \cdot \frac{2}{\lambda^{3}}+2 \lambda b \cdot \frac{1}{\lambda^{2}}+\lambda b^{2} \cdot \frac{1}{\lambda} \\
& =\frac{2}{\lambda^{2}}+\frac{2 b}{\lambda}+b^{2} \tag{7}
\end{align*}
\]

Replacing expression (2) and (7) into expression (1):

\[
\begin{align*}
\sigma_{x} & =\sqrt{\left(\frac{2}{\lambda^{2}}+\frac{2 b}{\lambda}+b^{2}\right)-\left(\frac{1}{\lambda}+b\right)^{2}} \\
& =\sqrt{\left(\frac{2}{\lambda^{2}}+\frac{2 b}{\lambda}+b^{2}\right)-\left(\frac{1}{\lambda^{2}}+\frac{2 b}{\lambda}+b^{2}\right)} \\
& =\frac{1}{\lambda}
\end{align*}
\]

\subsubsection{\texorpdfstring{In conclusion, the standard deviation of
\(X\) is
\(\sigma_{x}=\frac{1}{\lambda}\).}{In conclusion, the standard deviation of X is \textbackslash sigma\_\{x\}=\textbackslash frac\{1\}\{\textbackslash lambda\}.}}\label{in-conclusion-the-standard-deviation-of-x-is-sigma_xfrac1lambda.}

\paragraph{(3) i) Cumulative distribution function (CDF): Considering
the definition of CDF, it follows
that:}\label{i-cumulative-distribution-function-cdf-considering-the-definition-of-cdf-it-follows-that}

\[
\begin{equation*}
F_{x}(x)=P(x \leqslant x)=\int_{-\infty}^{\infty} p_{x}(t) d t . \tag{1}
\end{equation*}
\]

If \(x<b\), then \(p_{x}(t)=0\), thus there is:

\[
\begin{equation*}
F_{X}(x)=\int_{-\infty}^{x} 0 d t=0 \tag{2}
\end{equation*}
\]

If \(x \geqslant b\), then \(p_{x}(t)=\lambda e^{-\lambda(t-b)}\), thus
there is:

\[
\begin{equation*}
F_{X}(x)=\int_{b}^{x} \lambda e^{-\lambda(t-b)} d t \tag{3}
\end{equation*}
\]

Assuming that \(a=t-b\), then \(t=u+b\). and \(d t=d u\), the upper and
Cower limit of integration becomes \(0 \rightarrow x-b\).

\[
\begin{aligned}
F_{X}(x) & =\int_{0}^{x-b} \lambda e^{-\lambda u} d u \\
& =\lambda \cdot \int_{0}^{x-b} e^{-\lambda u} d u \\
& =\lambda\left[-\frac{1}{\lambda} e^{-\lambda u}\right]_{0}^{x-b} \\
& =\lambda \cdot\left(-\frac{1}{\lambda} e^{-\lambda(x-b)}+\frac{1}{\lambda} e^{0}\right) \\
& =\lambda \cdot\left(-\frac{1}{\lambda} e^{-\lambda(x-b)}+\frac{1}{\lambda}\right) \\
& =-e^{-\lambda(x-b)}+1 
\end{aligned}
\]

\subsubsection{\texorpdfstring{In conclusion, the cumulative
distribution function for the random variable \(x\) is
\(F_{x}(x)= \begin{cases}0, & \text { if } x<b \\ 1-e^{-x(x-b)}, & \text { if } x \geqslant b\end{cases}\)}{In conclusion, the cumulative distribution function for the random variable x is F\_\{x\}(x)= \textbackslash begin\{cases\}0, \& \textbackslash text \{ if \} x\textless b \textbackslash\textbackslash{} 1-e\^{}\{-x(x-b)\}, \& \textbackslash text \{ if \} x \textbackslash geqslant b\textbackslash end\{cases\}}}\label{in-conclusion-the-cumulative-distribution-function-for-the-random-variable-x-is-f_xx-begincases0-text-if-xb-1-e-xx-b-text-if-x-geqslant-bendcases}

\paragraph{ii) Quartile Function (QF)}\label{ii-quartile-function-qf}

Considering the CDF of random variable \(X\) :

\[
F_{X}(x)=\left\{\begin{array}{l}
0, \text { if } x<b  \tag{1}\\
1-e^{-\lambda(x-b),} \text { if } x \geqslant b
\end{array}\right.
\]

Quartile Function \(Q(p)\) satisfies:

\[
\begin{equation*}
F_{X}(Q(p))=p \text {, if } 0<p<1 \tag{2}
\end{equation*}
\]

Replacing (1) into cl), it follows that:

\[
\begin{equation*}
1-e^{-\lambda(x-b)}=p \tag{3}
\end{equation*}
\]

Thus there is:

\[
\begin{equation*}
e^{-\lambda(x-b)}=1-p \tag{4}
\end{equation*}
\]

Taking the natural logarithm of both sides:

\[
\begin{equation*}
-\lambda(x-b)=\ln (1-p) \tag{5}
\end{equation*}
\]

Thus \[
\begin{equation*}
x=b-\frac{\ln (1-p)}{\lambda} \tag{6}
\end{equation*}
\] \#\#\# In conclusion, the Quartile Function of random variable \(X\)
is \(Q(p)=b-\frac{\ln (1-p)}{\lambda}, 0<p<1\)

\paragraph{\texorpdfstring{(4) Maxmum Likehood estimat
\MLE:}{(4) Maxmum Likehood estimat :}}\label{maxmum-likehood-estimat}

The definition of likehood function is:

\[
\begin{align*}
L(\lambda) & =\prod_{i=1}^{n} p_{\lambda}\left(x_{i}\right) \\
& =\prod_{i=1}^{n} \lambda e^{-\lambda\left(x_{i}-b\right)} \\
& =\lambda^{n} \prod_{i=1}^{n} e^{-\lambda\left(x_{i}-b\right)} \\
& =\lambda^{n} e^{-\lambda \sum_{i=1}^{n}\left(x_{i}-b\right)} \tag{1}
\end{align*}
\]

Taking the natural logarithm of both sides:

\[
\begin{equation*}
\ln L(\lambda)=n \ln \lambda-\lambda \sum_{i=1}^{n}\left(x_{i}-b\right) \tag{2}
\end{equation*}
\]

Derivativing with respect to \(\lambda\) :

\[
\begin{equation*}
\frac{d \ln L(\lambda)}{d \lambda}=\frac{n}{\lambda}-\sum_{i=1}^{n}\left(x_{i}-b\right) \tag{3}
\end{equation*}
\]

Assuming that the derivative is zero, it follows that:

\[
\begin{equation*}
\frac{n}{\lambda}-\sum_{i=1}^{n}\left(x_{i}-b\right)=0 \tag{4}
\end{equation*}
\]

Calculating the value of \(\lambda\) is:

\[
\begin{equation*}
\lambda=\frac{n}{\sum_{i=1}^{n}\left(x_{i-b}\right)} \tag{5}
\end{equation*}
\]

\paragraph{\texorpdfstring{In conclusion, the maxmum likehood estimate
for \(x\) is
\(\lambda_{\text {MLE }}=\frac{n}{\sum_{i=1}^{n}\left(x_{i}-b\right)}\)}{In conclusion, the maxmum likehood estimate for x is \textbackslash lambda\_\{\textbackslash text \{MLE \}\}=\textbackslash frac\{n\}\{\textbackslash sum\_\{i=1\}\^{}\{n\}\textbackslash left(x\_\{i\}-b\textbackslash right)\}}}\label{in-conclusion-the-maxmum-likehood-estimate-for-x-is-lambda_text-mle-fracnsum_i1nleftx_i-bright}

\subsubsection{(5) Given the sample, compute and display the maximum
likelihood estimate λMLE of the parameter
λ.}\label{given-the-sample-compute-and-display-the-maximum-likelihood-estimate-ux3bbmle-of-the-parameter-ux3bb.}

\begin{Shaded}
\begin{Highlighting}[]
\CommentTok{\#Creating a lambda value calculation function}
\NormalTok{lambda\_function }\OtherTok{\textless{}{-}} \ControlFlowTok{function}\NormalTok{(data, b)\{}
\NormalTok{  n }\OtherTok{\textless{}{-}} \FunctionTok{length}\NormalTok{(data)}
  \CommentTok{\# Calculating the denominator: summing over x {-} b}
\NormalTok{  sum\_i }\OtherTok{\textless{}{-}} \FunctionTok{sum}\NormalTok{(data}\SpecialCharTok{{-}}\NormalTok{b)}
  \CommentTok{\# Calculating lembda}
\NormalTok{  lambda }\OtherTok{\textless{}{-}}\NormalTok{ n}\SpecialCharTok{/}\NormalTok{sum\_i}
  \FunctionTok{return}\NormalTok{(lambda)}
\NormalTok{\}}

\CommentTok{\# Load Data}
\NormalTok{supermarket\_data }\OtherTok{\textless{}{-}} \FunctionTok{read.csv}\NormalTok{(}\StringTok{"supermarket\_data\_2024.csv"}\NormalTok{)}
\CommentTok{\# Set the number of sample}
\NormalTok{n }\OtherTok{\textless{}{-}} \FunctionTok{nrow}\NormalTok{(supermarket\_data)}
\CommentTok{\# Set the constant b}
\NormalTok{b }\OtherTok{\textless{}{-}}\DecValTok{300}

\NormalTok{lambda\_mle }\OtherTok{\textless{}{-}} \FunctionTok{lambda\_function}\NormalTok{(supermarket\_data}\SpecialCharTok{$}\NormalTok{TimeLength , b)}
\FunctionTok{print}\NormalTok{(}\FunctionTok{paste}\NormalTok{(}\StringTok{"The maximum likelihood estimate lambda\_MLE is: "}\NormalTok{,lambda\_mle))}
\end{Highlighting}
\end{Shaded}

\begin{verbatim}
## [1] "The maximum likelihood estimate lambda_MLE is:  0.019884260798572"
\end{verbatim}

\subsubsection{(6) Compute the Bootstrap confidence
interval}\label{compute-the-bootstrap-confidence-interval}

\begin{Shaded}
\begin{Highlighting}[]
\CommentTok{\# set the random seed}
\FunctionTok{set.seed}\NormalTok{(}\DecValTok{2024}\NormalTok{)}
\CommentTok{\# set the number of resamples}
\NormalTok{number\_resamples }\OtherTok{\textless{}{-}} \DecValTok{10000}

\NormalTok{lambda\_resamples }\OtherTok{\textless{}{-}} \FunctionTok{numeric}\NormalTok{(number\_resamples)}
\CommentTok{\# Sampling with replacement}
\ControlFlowTok{for}\NormalTok{ (i }\ControlFlowTok{in} \DecValTok{1}\SpecialCharTok{:}\NormalTok{number\_resamples)\{}
\NormalTok{  sample\_data }\OtherTok{\textless{}{-}} \FunctionTok{sample}\NormalTok{(supermarket\_data}\SpecialCharTok{$}\NormalTok{TimeLength , }\AttributeTok{replace =} \ConstantTok{TRUE}\NormalTok{)}
\NormalTok{  lambda\_resamples[i] }\OtherTok{\textless{}{-}} \FunctionTok{lambda\_function}\NormalTok{(sample\_data, b)}
\NormalTok{\}}

\CommentTok{\#Calculate the 95\% confidence interval}
\NormalTok{  ci\_lower }\OtherTok{\textless{}{-}} \FunctionTok{quantile}\NormalTok{(lambda\_resamples, }\FloatTok{0.025}\NormalTok{)}
\NormalTok{  ci\_upper }\OtherTok{\textless{}{-}} \FunctionTok{quantile}\NormalTok{(lambda\_resamples, }\FloatTok{0.975}\NormalTok{)}
  \FunctionTok{print}\NormalTok{(}\FunctionTok{paste}\NormalTok{(}\StringTok{"The Bootstrap confidence interval is: ["}\NormalTok{,ci\_lower,}\StringTok{\textquotesingle{},\textquotesingle{}}\NormalTok{,ci\_upper,}\StringTok{\textquotesingle{}]}\SpecialCharTok{\textbackslash{}n}\StringTok{\textquotesingle{}}\NormalTok{))}
\end{Highlighting}
\end{Shaded}

\begin{verbatim}
## [1] "The Bootstrap confidence interval is: [ 0.0191115894121709 , 0.0206792527343937 ]\n"
\end{verbatim}

\subsubsection{(7)) Conduct a simulation study to explore the behaviour
of the maximum likelihood estimator λMLE for λ on simulated data X1, · ·
· ,
Xn}\label{conduct-a-simulation-study-to-explore-the-behaviour-of-the-maximum-likelihood-estimator-ux3bbmle-for-ux3bb-on-simulated-data-x1-xn}

\begin{Shaded}
\begin{Highlighting}[]
\CommentTok{\# Simulation parameters}
\FunctionTok{set.seed}\NormalTok{(}\DecValTok{2024}\NormalTok{)}
\NormalTok{lambda\_simulation }\OtherTok{\textless{}{-}} \DecValTok{2}
\NormalTok{b\_simulation }\OtherTok{\textless{}{-}} \FloatTok{0.01}
\NormalTok{sampleSize\_simulation }\OtherTok{\textless{}{-}} \FunctionTok{seq}\NormalTok{(}\DecValTok{100}\NormalTok{, }\DecValTok{5000}\NormalTok{, }\AttributeTok{by =} \DecValTok{10}\NormalTok{)}
\NormalTok{repetition\_simulation }\OtherTok{\textless{}{-}} \DecValTok{100}

\CommentTok{\# Simulation results}
\NormalTok{simulation\_results }\OtherTok{\textless{}{-}} \FunctionTok{data.frame}\NormalTok{(}\AttributeTok{SampleSize =}\NormalTok{ sampleSize\_simulation, }\AttributeTok{MSE =} \FunctionTok{numeric}\NormalTok{(}\FunctionTok{length}\NormalTok{(sampleSize\_simulation)))}


\CommentTok{\# Start simulation}
\NormalTok{index }\OtherTok{\textless{}{-}} \DecValTok{1}
\ControlFlowTok{for}\NormalTok{ (n }\ControlFlowTok{in}\NormalTok{ sampleSize\_simulation)\{}
\NormalTok{  lambda\_estimates}\OtherTok{\textless{}{-}}\FunctionTok{numeric}\NormalTok{(repetition\_simulation)}
  \CommentTok{\#For each sample size, consider 100 trials.}
  \ControlFlowTok{for}\NormalTok{(i }\ControlFlowTok{in} \DecValTok{1}\SpecialCharTok{:}\NormalTok{repetition\_simulation)\{}
\NormalTok{    sample\_data }\OtherTok{\textless{}{-}} \FunctionTok{rexp}\NormalTok{(n, }\AttributeTok{rate =}\NormalTok{ lambda\_simulation) }\SpecialCharTok{+}\NormalTok{ b\_simulation}
\NormalTok{    lambda\_estimates[i] }\OtherTok{\textless{}{-}} \FunctionTok{lambda\_function}\NormalTok{(sample\_data,b\_simulation)}
\NormalTok{  \}}
  \CommentTok{\#Calculate mean and Mean Squared Error(MSE)}
\NormalTok{  mse }\OtherTok{\textless{}{-}} \FunctionTok{mean}\NormalTok{((lambda\_estimates}\SpecialCharTok{{-}}\NormalTok{lambda\_simulation)}\SpecialCharTok{\^{}}\DecValTok{2}\NormalTok{)}
\NormalTok{  simulation\_results[index, }\StringTok{"MSE"}\NormalTok{] }\OtherTok{\textless{}{-}}\NormalTok{ mse}
\NormalTok{  index }\OtherTok{\textless{}{-}}\NormalTok{ index}\SpecialCharTok{+}\DecValTok{1}
\NormalTok{\}}

\CommentTok{\#Check the result}
\FunctionTok{head}\NormalTok{(simulation\_results,}\DecValTok{5}\NormalTok{)}
\end{Highlighting}
\end{Shaded}

\begin{verbatim}
##   SampleSize        MSE
## 1        100 0.04507438
## 2        110 0.03664986
## 3        120 0.03907616
## 4        130 0.03197187
## 5        140 0.02650689
\end{verbatim}

\begin{Shaded}
\begin{Highlighting}[]
\CommentTok{\#Plotting results of the simulation}
\FunctionTok{ggplot}\NormalTok{(simulation\_results,}
       \FunctionTok{aes}\NormalTok{(}\AttributeTok{x=}\NormalTok{SampleSize, }\AttributeTok{y=}\NormalTok{MSE))}\SpecialCharTok{+}
  \FunctionTok{geom\_line}\NormalTok{() }\SpecialCharTok{+}
  \FunctionTok{labs}\NormalTok{(}\AttributeTok{x=}\StringTok{"Sample Size"}\NormalTok{, }\AttributeTok{y=}\StringTok{"Mean Squared Error"}\NormalTok{)}
\end{Highlighting}
\end{Shaded}

\includegraphics{TaskB_files/figure-latex/unnamed-chunk-3-1.pdf}

\subsection{B2}\label{b2}

\subsubsection{(1) The formula for the probability mass
function}\label{the-formula-for-the-probability-mass-function}

The range of possible values of the discrete ranom variable \(x\) is:

\[
x \in\{-2,0,2\}
\]

Event 1: Draw 2 blue balls \((X=-2)\)

\[
\begin{aligned}
& C(b, 2)=\frac{b(b-1)}{2}, C(a+b, 2)=\frac{(a+b)(a+b-1)}{2} \\
& \therefore P(x=-2)=\frac{C(b, 2)}{C(a+b, 2)}=\frac{b(b-1)}{(a+b)(a+b-1)}
\end{aligned}
\]

Event 2: Draw 1 red ball and 1 blue ball \((x=0)\)

\[
\begin{aligned}
& C(a, 1) \cdot C(b, 1)=a \cdot b \\
\therefore  & P(x=0)=\frac{C(a, 1) C(b, 1)}{C(a+b, 2)}=\frac{a b}{(a+b)(a+b-1)}
\end{aligned}
\]

Event 3: Draw 2 red balls \((x=2)\)

\[
\begin{aligned}
C(a, 2) & =\frac{a(a-1)}{2} \\
\therefore P(x=2) & =\frac{C(a, 2)}{C(a+b, 2)}=\frac{a(a-1)}{(a+b)(a+b-1)}
\end{aligned}
\]

\subsubsection{In conclustion, the formula for the probability mass
function
pros:}\label{in-conclustion-the-formula-for-the-probability-mass-function-pros}

\[
P(x)= \begin{cases}\frac{b(b-1)}{(a+b)(a+b-1)} & , x=-2 \\ \frac{a b}{(a+b)(a+b-1)} & , x=0 \\ \frac{a(a-1)}{(a+b)(a+b-1)} & , x=2\end{cases}
\]

\subsubsection{(2) The expression of the
expectation}\label{the-expression-of-the-expectation}

According to the expectation formula for random variables:

\[
\begin{equation*}
E(x)=\sum_{x \in\{-2,0,2\}} x \cdot p(x=x) \tag{1}
\end{equation*}
\]

Replacing \(P(x)\) into (1) , there is:

\[
\begin{align*}
E(x) & =-2 \cdot \frac{b(b-1)}{(a+b)(a+b-1)}+2 \cdot \frac{a(a-1)}{(a+b)(a+b-1)} \\
& =\frac{2 a(a-1)-2 b(b-1)}{(a+b)(a+b-1)} \\
& =\frac{2\left(a^{2}-b^{2}-a+b\right)}{(a+b)(a+b-1)} \tag{2}
\end{align*}
\]

\subsubsection{\texorpdfstring{In conclusion, the expectation \(E(X)\)
of \(X\)
is:}{In conclusion, the expectation E(X) of X is:}}\label{in-conclusion-the-expectation-ex-of-x-is}

\[
E(X)=\frac{2\left(a^{2}-b^{2}-a+b\right)}{(a+b)(a+b-1)}
\]

\subsubsection{(3)The expression of the variance
Var(X)}\label{the-expression-of-the-variance-varx}

According to the formula of \(E\left(X^{2}\right)\) :

\[
\begin{equation*}
E\left(x^{2}\right)=\int_{x \in\{-10,2\}} x^{2} \cdot P(x=x) \tag{1}
\end{equation*}
\]

Replacing \(P(x)\) in to \((1)\), there is:

\[
\begin{align*}
E\left(x^{2}\right) & =4 \cdot \frac{b(b-1)}{(a+b)(a+b-1)}+4 \cdot \frac{a(a-1)}{(a+b)(a+b-1)} \\
& =\frac{4\left(a^{2}+b^{2}-a-b\right)}{(a+b)(a+b-1)} \tag{2}
\end{align*}
\]

According to the formula of \(\operatorname{Var}(X)\) :

\[
\begin{equation*}
\operatorname{Var}(x)=E\left(x^{2}\right)-[E(x)]^{2} \tag{3}
\end{equation*}
\]

Replacing \(E\left(x^{2}\right)\) and \(E(x)\) into (3), it follows
that:

\[
\begin{aligned}
\operatorname{Var}(x) & =\frac{4\left(a^{2}+b^{2}-a-b\right)}{(a+b)(a+b-1)}-\left[\frac{2\left(a^{2}-b^{2}-a+b\right)}{(a+b)(a+b-1)}\right]^{2} \\
& =\frac{4\left(a^{2}+b^{2}-a-b\right)(a+b)(a+b-1)-4\left(a^{2}-b^{2}-a+b\right)^{2}}{(a+b)^{2}(a+b-1)^{2}}
\end{aligned}
\] \#\#\# In conclusion, the expression of the variance Var(X) is:\\
\[
\begin{aligned}
\operatorname{Var}(x) & =\frac{4\left(a^{2}+b^{2}-a-b\right)(a+b)(a+b-1)-4\left(a^{2}-b^{2}-a+b\right)^{2}}{(a+b)^{2}(a+b-1)^{2}}
\end{aligned}
\] \#\#\# (4)Write a function called compute\_expectation\_X that takes
a and b as inputs and outputs the expectation E(X). Write a function
called compute\_variance\_X that takes a and b as input and outputs the
variance Var(X).

\begin{Shaded}
\begin{Highlighting}[]
\CommentTok{\# The function of E(X)}
\NormalTok{compute\_expectation\_X }\OtherTok{\textless{}{-}} \ControlFlowTok{function}\NormalTok{(a,b)\{}
\NormalTok{  f1 }\OtherTok{\textless{}{-}} \DecValTok{2}\SpecialCharTok{*}\NormalTok{(a}\SpecialCharTok{\^{}}\DecValTok{2}\SpecialCharTok{{-}}\NormalTok{b}\SpecialCharTok{\^{}}\DecValTok{2}\SpecialCharTok{{-}}\NormalTok{a}\SpecialCharTok{+}\NormalTok{b)}
\NormalTok{  f2 }\OtherTok{\textless{}{-}}\NormalTok{ (a}\SpecialCharTok{+}\NormalTok{b)}\SpecialCharTok{*}\NormalTok{(a}\SpecialCharTok{+}\NormalTok{b}\DecValTok{{-}1}\NormalTok{)}
\NormalTok{  expectation }\OtherTok{=}\NormalTok{ f1}\SpecialCharTok{/}\NormalTok{f2}
  \FunctionTok{return}\NormalTok{(expectation)}
\NormalTok{\}}

\CommentTok{\# The function of Var(x)}
\NormalTok{compute\_variance\_X }\OtherTok{\textless{}{-}} \ControlFlowTok{function}\NormalTok{(a,b)\{}
  \CommentTok{\#f1 \textless{}{-} (a + b) * (a + b {-} 1)                }
  \CommentTok{\#f2 \textless{}{-} 4 * (a * (a {-} 1) + b * (b {-} 1)) / f1  }
  \CommentTok{\#f3 \textless{}{-} 4 * (a\^{}2 {-} b\^{}2 {-} a + b)\^{}2 / f1\^{}2  }
  \CommentTok{\#variance \textless{}{-} f2{-}f3           }
\NormalTok{  f1 }\OtherTok{\textless{}{-}}\NormalTok{ (a}\SpecialCharTok{+}\NormalTok{b)}\SpecialCharTok{*}\NormalTok{(a}\SpecialCharTok{+}\NormalTok{b}\DecValTok{{-}1}\NormalTok{)}
\NormalTok{  f2 }\OtherTok{\textless{}{-}}\NormalTok{ (a}\SpecialCharTok{\^{}}\DecValTok{2}\SpecialCharTok{+}\NormalTok{b}\SpecialCharTok{\^{}}\DecValTok{2}\SpecialCharTok{{-}}\NormalTok{a}\SpecialCharTok{{-}}\NormalTok{b)}\SpecialCharTok{*}\NormalTok{f1}
\NormalTok{  f3 }\OtherTok{\textless{}{-}}\NormalTok{ (a}\SpecialCharTok{\^{}}\DecValTok{2}\SpecialCharTok{{-}}\NormalTok{b}\SpecialCharTok{\^{}}\DecValTok{2}\SpecialCharTok{{-}}\NormalTok{a}\SpecialCharTok{+}\NormalTok{b)}\SpecialCharTok{\^{}}\DecValTok{2}
\NormalTok{  variance }\OtherTok{\textless{}{-}}\NormalTok{ (}\DecValTok{4}\SpecialCharTok{*}\NormalTok{f2}\DecValTok{{-}4}\SpecialCharTok{*}\NormalTok{f3)}\SpecialCharTok{/}\NormalTok{f1}\SpecialCharTok{\^{}}\DecValTok{2}
  \FunctionTok{return}\NormalTok{(variance)}
\NormalTok{\}}
\end{Highlighting}
\end{Shaded}

\subsubsection{(5) According to the linear nature of the mean of a
random
variable:}\label{according-to-the-linear-nature-of-the-mean-of-a-random-variable}

\[
\begin{equation*}
E(\bar{X})=E\left(\frac{1}{n} \sum_{i=1}^{n} X_{i}\right)=\frac{1}{n} \sum_{i=1}^{n} E\left(X_{i}\right) \tag{1}
\end{equation*}
\]

\(\because X_{i}\) are i.i.d., thus \(E\left(X_{i}\right)=E(x)\), thus:

\[
\begin{equation*}
E(\bar{X})=\frac{1}{n} \cdot n \cdot E(X)=E(X) \tag{2}
\end{equation*}
\]

\subsubsection{\texorpdfstring{In conclusion, the expression of the
expectation of the random variable \(\bar{x}\) is:
\(E(\bar{x})=\frac{2\left(a^{2}-b^{2}-a+b\right)}{(a+b)(a+b-1)}\)}{In conclusion, the expression of the expectation of the random variable \textbackslash bar\{x\} is: E(\textbackslash bar\{x\})=\textbackslash frac\{2\textbackslash left(a\^{}\{2\}-b\^{}\{2\}-a+b\textbackslash right)\}\{(a+b)(a+b-1)\}}}\label{in-conclusion-the-expression-of-the-expectation-of-the-random-variable-barx-is-ebarxfrac2lefta2-b2-abrightabab-1}

\#\#\#(6) Considering the effect of a linear transformation of a random
variable on the variance :

\[
\begin{equation*}
& \operatorname{Var}(a x+b)=a^{2} \operatorname{Var}(x)  \tag{1}\\
\end{equation*}
\] thus: \[
\begin{align*}
&\operatorname{Var}(\bar{x})=\operatorname{Var}\left(\frac{1}{n} \sum_{i=1}^{n} X_{i}\right) \\
&=\frac{1}{n^{2}} \operatorname{Var}\left(\sum_{i=1}^{n} X_{i}\right) \\
&=\frac{1}{n^{2}} \cdot n \operatorname{Var}(X) \\
&=\frac{1}{n} \operatorname{Var}(X) \tag{2}
\end{align*}
\]

\subsubsection{\texorpdfstring{In conclusion, the expression of the
variance of the random variable \(\bar{x}\)
is}{In conclusion, the expression of the variance of the random variable \textbackslash bar\{x\} is}}\label{in-conclusion-the-expression-of-the-variance-of-the-random-variable-barx-is}

\[
\operatorname{Var}(\bar{X})=\frac{4\left(a^{2}+b^{2}-a-b\right)(a+b)(a+b-1)-4\left(a^{2}-b^{2}-a+b\right)^{2}}{n(a+b)^{2}(a+b-1)^{2}}
\]

\end{document}
